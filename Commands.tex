\newcommand{\finn}{\vspace{1.5ex}}

\newcommand{\fix}{\vspace{-23pt}} %reduce wasted space

%Easy notation of physical units
\newcommand{\siu}[1]{[\si{#1}]}

\newcommand{\Eofr}{\vec{E}\left(\vec{r}\right)} 	%Elektrisches Feld

\newcommand{\Bofr}{\vec{B}\left(\vec{r}\right)}	%Magnetisches Feld

\newcommand{\Hofr}{\vec{H}\left(\vec{r}\right)}	%Magnetische Erregung

\newcommand{\Fofr}{\vec{F}\left(\vec{r}\right)}	%Kraftfeld

\newcommand{\Vprod}[2]{\vec{#1}\times\vec{#2}} 	%Vektorprodukt

\newcommand{\rng}{\text{\textsc{Range}}}

\newcommand{\spn}{\text{\textsc{Span}}}

\newcommand{\nll}{\text{\textsc{Null}}}

\newcommand{\dahe}{$\rightarrow$ }	%easy arrow outside mathmode

\newcommand{\compaq}{\setlength{\itemsep}{-1mm}\setlength{\parskip}{0cm}}%compact itemizes

\newcommand{\ncompaq}{\setlength{\itemsep}{1mm}\setlength{\parskip}{0cm}}%compact itemizes

\newcommand{\mypic}[1]{\includegraphics[width=\linewidth]{#1}} %easy including of pictures within the column

\newcommand{\myspic}[2]{\begin{center}\includegraphics[width=#1\linewidth]{#2}\end{center}} %more uneasy including of pictures

\newcommand{\hfull}{\hfill$|$\hfill} %easy seperation of two equations
%end commands

\newcommand{\longeq}{\hfill$=$\hfill}%easy seperation of an equation

\newcommand{\intinf}[1]{\int_{-\infty}^\infty{#1}}

\newcommand{\expval}[1]{\langle #1\rangle}

\newcommand{\important}[1]{\begin{center}\fbox{#1}\end{center}}

\newcommand{\importname}[2]{\begin{center}\fbox{#2}  #1\end{center}}

\newcommand{\mimportant}[1]{\begin{center}\begin{tabular}{|c|}\hline #1 \\ \hline\end{tabular}\end{center}}

\newcommand{\importable}[1]{\begin{center}\begin{tabular}{|ll|}\hline #1 \\ \hline\end{tabular}\end{center}}

\newcommand{\importabflex}[2]{\begin{center}\begin{tabular}{|#1|}\hline #2\\ \hline\end{tabular}\end{center}}

\newcommand{\mportant}[1]{\begin{center}#1\end{center}}

\newcommand{\mportname}[2]{\begin{center}#2$\quad$#1\end{center}}

\newcommand{\mmportant}[1]{\begin{center}\begin{tabular}{c} #1 \end{tabular}\end{center}}

\newcommand{\mportable}[1]{\begin{center}\begin{tabular}{ll}#1\end{tabular}\end{center}}

\newcommand{\mportabflex}[2]{\begin{center}\begin{tabular}{#1}#2\end{tabular}\end{center}}

\newcommand{\note}[1]{\footnotesize #1 \normalsize}

\newcommand{\notelist}[1]{\footnotesize \begin{itemize}\compaq #1\end{itemize}\normalsize}

\newcommand{\lstfill}{\vspace{4ex}}

\newcommand{\inp}[2]{\langle #1 | #2 \rangle} % fast inner product

\newcommand{\bra}[1]{\langle #1 |}

\newcommand{\ket}[1]{| #1 \rangle}

\newenvironment{brsm}{% % short for 'bracketed small matrix'
  \bigl( \begin{smallmatrix} }{\end{smallmatrix} \bigr)}

\newcommand{\vectwo}[2]{\begin{brsm}#1 \\ #2\end{brsm}}

\newcommand{\element}[2]{\in\mathbb{R}^{#1 \times #2}}

\newcommand{\dive}{\vec{\nabla}\cdot}

\newcommand{\rota}{\vec{\nabla}\times}

\newcommand{\ddt}{\frac{d}{dt}}

\newcommand{\onha}{\frac{1}{2}}

\DeclareRobustCommand{\hlcyan}[1]{{\sethlcolor{Aquamarine1}\hl{#1}}}

\DeclareRobustCommand{\hlyellow}[1]{{\sethlcolor{Yellow1}\hl{#1}}}

\DeclareRobustCommand{\hlpink}[1]{{\sethlcolor{Orchid1}\hl{#1}}}

\DeclareRobustCommand{\hlgreen}[1]{{\sethlcolor{PaleGreen1}\hl{#1}}}

\DeclarePairedDelimiter\ceil{\lceil}{\rceil}

\DeclarePairedDelimiter\floor{\lfloor}{\rfloor}

\newcommand{\prob}[1]{\mathbb{P}\left(#1\right)}

\newcommand{\expe}[1]{\mathbb{E}\left[#1\right]}

\newcommand{\var}[1]{\text{Var}\left(#1\right)}

\newcommand{\cov}[1]{\text{Cov}\left(#1\right)}

\newcommand{\corr}[1]{\text{Corr}\left(#1\right)}

\newcommand{\sign}[1]{\text{sign}\left(#1\right)}

\newcommand{\diag}{\text{diag}}

\newcommand{\vtabfill}{\vspace{-0.7ex}$\ $}

\newcommand{\lap}[1]{\mathcal{L}(#1)}

\newcommand{\sinc}{\text{sinc}}

\newcommand{\sbs}[4]{\begin{minipage}[t!]{#1\linewidth}#3\end{minipage}\begin{minipage}[t!]{#2\linewidth}#4\end{minipage}}

\newcommand{\sbss}[2]{\sbs{0.45}{0.45}{#1}{#2}}

\renewcommand{\vec}[1]{\underline{#1}}

\newcommand{\tvec}[1]{\underline{\underline{#1}}}

\newcommand{\vecd}[1]{\underline{\dot{#1}}}

\newcommand{\vecdd}[1]{\underline{\ddot{#1}}}

\newcommand{\easyurl}[2]{\href{#2}{\textcolor{blue}{\underline{#1}}}}

\newcommand{\definitiontable}[1]{
\note{
\begin{center}
\begin{tabular}{l@{$\quad$}l@{\dotfill$\quad$}cl}
#1
\end{tabular}
\end{center}
}}

\newcommand{\highlight}[1]{%
  \colorbox{cyan!20}{$\displaystyle#1$}}

%Setup of code-snippet inclusion with \usepackage{listings}
\newcommand{\lsgeneral}{
\lstset{basicstyle=\ttfamily,
breaklines=true,%
showstringspaces=false,%without this there will be a symbol in the places where there is a space
showspaces=false,% show spaces adding particular underscores
showtabs=false, % show tabs within strings adding particular underscores
tabsize=2,
breaklines=true,
numbers=left,%
numberstyle={\tiny \color{black}},% size of the numbers
numbersep=9pt, % this defines how far the numbers are from the text
frame=single,
breakatwhitespace=false,% sets if automatic breaks should only happen at whitespace
xleftmargin=0.05\textwidth,
columns=fixed,
captionpos=b,
}
}
  

%Definition of the style for cpp code
\newcommand{\lscpp}{
\lstset{ 
	language=C++,% choose the language of the code
	backgroundcolor=\color{black!10!white},
	basicstyle=\color{black}\ttfamily,
    keywordstyle=\color{violet!80!black},
    stringstyle=\color{red!80!black},
    commentstyle=\color{green!80!black},
    morecomment=[l][\color{blue!80!black}]{\#}
}
}

%Definition of the environment encapsulating cpp code
\lstnewenvironment{TPCpp}[1][]{
	\vspace{3ex}
	\lstset{caption={#1}}
	\lsgeneral
	\lscpp
	\centering\minipage{.96\linewidth}%
}{
	\endminipage%
	\vspace{1ex}
}

%Definition of the style for matlab code
\newcommand{\lsmatlab}{
\lstset{language=Matlab,
keywordstyle=\color{blue},%
keywordstyle=[2]{\color{black}},
identifierstyle=\color{black},%
stringstyle=\color{blue!20!violet},
commentstyle=\color{black!30!green},
backgroundcolor=\color{yellow!20!white},
}
}

%Definition of the environment encapsulating matlab code
\lstnewenvironment{TPMatlab}[1][]{
	\vspace{3ex}
	\lstset{caption={#1}}
	\lsgeneral
	\lsmatlab
	\centering\minipage{.96\linewidth}%
}{
	\endminipage%
	\vspace{3ex}
}
\NewDocumentEnvironment{TDefinitionTable*}{} %Used to define variables in an equation
{
%\rowcolors{1}{gray!25}{white}
\footnotesize
\begin{center}
\begin{tabular}{lll}
}
{
\end{tabular}
\end{center}
\normalsize
}
%Usage:
%\begin{TDefinitionTable}
%$\vect{P}$&change of linear momentum &in& $\si{\kilo\gram\meter\per\second\squared}$\\
%$m$&mass&in&$\si{\kilo\gram}$\\
%$\vec{a}$&acceleration&in&$\si{\meter\per\second\squared}$\\
%$\vec{F}$&resultant force&in&$\si{N}=\si{\kilo\gram\meter\per\second\squared}$
%\end{TDefinitionTable}


